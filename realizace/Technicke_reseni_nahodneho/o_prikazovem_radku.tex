\label{sec:o_prikazovem_radku}

Výše zmíněný přikazový řádek je jedním z prvků, kterým lze počítač ovládat. Lze ho spustit kliknutím na ikonu na hlavním liště počítače. Hráč je částečně s jeho fungováním seznámen již v sekci výuky. V této sekci je popsána jeho interakce s uživatelem a následně několik dalších zůsobů, kterými může uživatel s virtuálním počítačem v aktuální verzi hry komunikovat.

Řádek simuluje pouze několik nejčastějších a nejjednoduších příkazů. Rozcestníkem pro tyto příkazy je stejně jako v opravdovém příkazovém řádku příkaz \textit{help}. Výstupem příkazu \textit{help} je seznam všech příkazů, které jsou momentálně dostupné. Patří mezi ně například: cd, dir, del, getmac, ipconfig, netstat, aj.

Dále počítačové rozhraní obsahuje generované soubory jejichž generace je popsána v~kapitole \ref{sec:generace_rozhrani}. Tyto soubory se skládají z ikony, názvu a přípony. Každý z těchto prvků může indikovat nebezpečný soubor. Je tak učiněno očividným způsobem.