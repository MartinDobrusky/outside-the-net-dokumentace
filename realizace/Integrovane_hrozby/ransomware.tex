\subsection{Ransomware}

Ransomware je škodlivý software, který blokuje přístup k počítačovému systému nebo šifruje data a vyžaduje výkupné za jejich obnovení. Tento typ útoku je obvykle spuštěn kliknutím na odkaz nebo stažením souboru z podezřelého zdroje. Jakmile se ransomware dostane do počítačového systému, začne šifrovat data a zobrazuje výzvu k platbě výkupného za obnovení dat. Útočník obvykle požaduje platbu v kryptoměně, aby bylo obtížnější jeho vystopování.

Ransomware se může šířit prostřednictvím e-mailových příloh, škodlivých reklam nebo prostřednictvím nezabezpečených sítí. Ochrana proti ransomware zahrnuje pravidelné zálohování dat, instalaci aktualizací softwaru a firewally, používání silných hesel a pozornost při stahování souborů a klikání na odkazy z neznámých zdrojů.

Pokud se stane, že váš počítač je napaden ransomware, nedoporučuje se platit výkupné, protože to nezaručuje, že získáte svá data zpět a může motivovat útočníky k dalšímu útoku.

Více detailních inforamcí je obsaženo v knize Thread Landsacpe 2022 na stranách 43 až 48. \cite{enisa}