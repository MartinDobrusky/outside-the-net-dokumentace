\subsection{Malware}

První z integrovaných hrozeb je nejčastější hrozbou dnešní doby.

Malware je zkratka pro "malicious software", což v překladu znamená "škodlivý software". Jedná se o jakýkoli typ softwaru, který byl navržen tak, aby způsobil škodu na počítači, síti nebo na datech, která jsou na těchto systémech uložena.

Existuje mnoho druhů malware, včetně virů, trojanů, spyware, adware, ransomware a~dalších. Tyto programy se mohou šířit různými způsoby, například pomocí emailových příloh, souborů ke stažení z internetu, infikovaných webových stránek nebo pomocí externích paměťových zařízení.

Malware může mít různé cíle, včetně krádeže osobních údajů, poškození systému, šíření spamu, zobrazování reklam nebo blokování přístupu k datům a požadování výkupného. Může také být použit k infikování dalších počítačů v síti nebo k vytvoření botnetu, což je síť počítačů infikovaných malwarem, která může být ovládána z jednoho centrálního místa.

Ochrana proti malware zahrnuje instalaci antivirového software, aktualizace operačního systému a aplikací, používání silných hesel a opatrnost při otevírání neznámých souborů nebo klikání na neznámé odkazy.

Více detailních inforamcí je obsaženo v knize Thread Landsacpe 2022 na stranách 49 až 53. \cite{enisa}