\subsubsection{Sekce náhodně generovaného scénáře}
\label{sec:nahodny_scenar}

Sekce náhodně generovaného scénáře je první a primární herní sekcí.

Po spuštění je hráč přenesen na herní mapu. Herní mapa je tvořena jednopodlažním kancelářským komplexem. Tato mapa je vygenerována za pomocí algoritmu náhodného rozmístění objektů, blíže popsaného v kapitole \ref{sec:finalni_provedeni_mapy}. Tato generace slouží k umožnění repetitivního hraní tohoto scénáře. Vzniká tak pro hráče možnost čelit modifikovaným rozložením a tehdy i možnost své znalosti ověřit a zdokonalit opakovaným hraním. Zároveň také tento fakt přispívá k zábavnosti celého herního módu.

Hráčova kamera zabírá pohled shora na tento komplex. Hráč má možnost volného pohybu nad celým komplexem a zároveň otáčení pohledu kamery dle libosti. Kamera se dá teké přiblížit a oddálit.

Po mapě jsou rozmístěny počítačové stanice. Hráč má možnost s těmito stanicemi interagovat stisknutím levého tlačítka myši. Po stisknutí tlačítka je otevřeno počítačové rozhraní. Toto rozhraní je taktéž náhodně generováno, pro dokreslení již zmiňovaného efektu. Tento proces je více popsán v kapitole \ref{sec:generace_rozhrani}. Rozhraní obsahuje všechny očekávané prvky včetně pozadí, souborů a hlavní lišty.

Posledním hlavním prvkem je simulovaný příkazový řádek. Tento příkazový řádek imituje funkcionality několika nejběžnějších příkazů. Hráč se tedy s těmito příkazy seznámí a následně je využije při řešení problematiky hrozeb.