\subsection{Nastavení}
\label{sec:nastaveni}
Poslední obrazovkou je obrazovka nastavení. Tato obrazovka je stejně jako všechny ostatní integrována do úvodní scény a je zobrazena po animaci přechodu kamery. Obsahuje několik nastavení, která jsou přímo propisována do zbytku hry a ovlivňují vstupní parametry v~náhodném herním scénáři.

První z nastavení je zobrazení FPS. Tento ukazatel je vypočítáván přímo ve hře v~závislosti na opravdovém času vykreslování. Slouží hráči k lepší orientaci týkající se výkonu hry a může pomoci při výběru vhodných grafických nastavení.

Druhým nastavením je nastavení obtížnosti hry, toto nastavení se projevuje při generaci mapy i při následném průběhu hry. Čím těžší je obtížnost, tím více počítačových stanic mapa obsahuje. Následně jsou během hry přidělovány hrozby v intervalu odpovídajícím obtížnosti hry. Těžší nastavení opět odpovídá větší intenzitě zobrazovaných problémů.

Dalším nastavením je celková hlasitost hry. Toto nastavení ovlivňuje celkovou hlasitost hry, včetně hudby a zvuků. Interaktivní posuvník označený popiskem Master Volume vyjadřuje porcentuální hlasitost vůči nejvyšší hlasitosti OS.

V neposlední řadě je několik nastavení věnováno grafickému zobrazení. Tato nastavení mají za úkol zajistit správné zobrazení a co nejplynulejší běh hry. Rozlišení mění rozlišení obrazu v pixelech. Grafická nastavení jsou zaobalena do přehledných nabídek v rozbalovacím okně. Na pozadí dochází k výměně Universal Renering Pipeline objektů v nastavení URP. Tyto objekty mění například rozlišení textur, chovaní a kvalitu stínů nebo zobrazení a intenzitu postprocessingu. Všechny tyto nastavení jsou výchozí a jsou přepsány při spuštění hry podle posledních vybraných nastavení.