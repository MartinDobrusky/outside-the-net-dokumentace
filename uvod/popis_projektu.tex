\chapter{Popis projektu}

Projekt Outside The Net je počítačová videohra zaměřená na ukázku vybraných kybernetických hrozeb. Hra je rozdělena do dvou hlavních hratelných sekcí.

První sekcí je sekce výuky. V této sekci je nejprve uživatel seznámen s fungováním a~ovládáním hry samotné. Poté jsou mu postupně představovány jednotlivé naimplementované hrozby. Součástí těchto prezentací jsou doprovodné texty s praktickou ukázkou. Taktéž je zde několik stručných popisů nejvíce problematických oblastí, mezi které patří například sociální inženýrství a s ním spojené techniky, problematika hesel a phising se zaměřením na e-mailovou komunikaci. Sekce je blíže popsána v sekci \ref{sec:vyuka_scenar}.

Druhou sekcí jsou náhodně generované scénáře obsahující některé hrozby představené v sekci výuky. Hráč se pohybuje v prostředí kancelářského komplexu s rozmístěnými uživatelskými stanicemi. Herní doba je rozdělena na dny o fiktivní časové délce, během kterých se postupně přidávají další a obtížnější překážky, kterým musí hráč čelit. Tato sekce je blíže popsána v sekci \ref{sec:nahodny_scenar}.

Vše je koncipováno klasickou videoherní formou, hra tedy obsahuje úvodní obrazovku a obrazovku nastavení. Úvodní obrazovka je stylizována do herního zasazení a uživatelské rozhraní se integrováno do 3D prostoru. Pro celý projekt byl zvolen jazyk angličtina, jelikož je tímto způsobem zajištěna dostupnost pro co největší možný počet uživatelů.

Pracovní název projektu je aktuálně Outside The Net. Cílem tohoto názvu je vyjádřit izolaci od počítačové/internetové sítě. Název by se dal tedy interpretovat jak ze strany osamoceného kancelářského komplexu, ve kterém se videohra odehrává, tak zároveň ze strany hackerů, kteří se snaží svým chováním zůstat taktéž mimo pomyslnou síť.

Projekt se aktuálně nachází ve své beta verzi. Během vývoje prošel několika alfa verzemi. Aktuální verze je tedy prozatím předběžná, jelikož je v plánu uzavřené veřejné testování. Cílem testování bude samotná hratelnost a taktéž výkon a optimalizace na různých operačních systémech. Po dokončení tohoto veřejného testování je v plánu vydat veřejně dostupnou verzi. Dle předběžných konzultací je taktéž možné, že výsledný produkt poslouží jako výukový nástroj.
