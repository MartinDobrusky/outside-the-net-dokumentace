\chapter*{Zadání maturitního projektu z informatických předmětů}
\begin{tabular}{l l}
    Jméno a příjmení: & Martin Dobruský \\
    Pro školní rok: & 2022/2023 \\
    Třída: & 4. \\
	Obor: & Informační technologie 18-20-M/01 \\\\

    Téma práce: & Návrh a tvorba hry Outside The Net \\
    Vedoucí práce: & Mgr. Josef Horálek, Ph.D. \\
\end{tabular} \\\\

Způsob zpracování, cíle práce, pokyny k obsahu a rozsahu práce: \\
Cílem maturitního projektu je navrhnout a vytvořit videohru Outside The Net, jejímž cílem bude formou hry seznámit hráče s principy etického hackingu a zabezpečení výpočetních a komunikačních systémů. Hra bude 3D a bude hráči nabízet přímou interakci s infrastrukturou sítě. Ve hře budou obsaženy i principy hackingu netechnicky založené, například metoda Phishing. Hráč bude tedy seznámen s nejčastějšími pojmy v oboru zabezpečení výpočetní techniky a komunikačních systémů. Autor práce navrhne logické části hry Outside The Net vybere nejčastější bezpečností hrozby a způsob jejich začlenění do hry. V praktické části autor realizuje návrh Outside The Net za využití herního enginu Unity. Pilotním jazykem práce bude jazyk C\#, který bude využit s důrazem na standardy OOP. Pro tvorbu 3D assetů bude využit software Blender. \\\\
Stručný časový harmonogram (s daty a konkretizovanými úkoly): \\
Září – Říjen 2022 \\
\hspace{1cm}Návrh principů hry Outside The Net \\
Listopad – Prosinec 2022 \\
\hspace{1cm}Definování, a realizace dílčích částí hry Outside The Net \\
Prosinec – Únor 2022/2023 \\
\hspace{1cm}Komplexní implementace navrženého řešení \\
Únor – Březen 2023 \\
\hspace{1cm}Dokončení praktických řešení \\
Březen 2023 \\
\hspace{1cm}Finalizace textového znění maturitního projektu \\
