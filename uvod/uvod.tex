Bezpečnost v on-line světě se stává čím dál zásadnějším tématem ve společnosti. Mladší generace využívají počítače i internet prakticky každodenně, a přesto si mnohdy nejsou vědomi všudypřítomného nebezpečí a negativních následků, které může způsobit. Je zřejmé, že snaha o kybernetickou osvětu celosvětově stoupá. Podrobná data týkající se například investic nebo pracovních příležitostí v tomto oboru jsou zpracována v průzkumu State of Cybersecurity 2022 Report od mezinárodního profesního sdružení ISACA \cite{isaca}. Momentálně se tato snaha asi nejvíce projevuje u firem, které nabízí služby v této oblasti, a~u~příslušných státních orgánů. Dle dat vyplývajících ze studie roku 2020 \cite{ita} pořádané Mezinárodní telekomunikační unií patří Česká republika celosvětově mezi vyspělejší země v oblasti kybernetické bezpečnosti. V současnosti u nás působí poměrně velké množství firem zabývajících se kybernetickou bezpečností a taktéž NUKIB (Národní úřad pro kybernetickou~a informační bezpečnost) je velmi aktivním orgánem, jenž přináší aktuální a~ověřené informace a doporučení. \cite{nukib}

Ve své práci se zaměřuji na vývoj 3D videohry s pracovním názvem Outside The Net, která si klade za cíl přiblížit problematiku kybernetické bezpečnosti hráčům. Samozřejmě není cílem postihnout celou výše zmíněnou oblast nebo přinést velmi detailní a technicky přesný vhled. Realizace takovéhoto projektu by vyžadovala tým odborníků, stabilní finanční zázemí a delší čas vývoje. Proto se projekt zaměřuje pouze na obecnou rovinu a~nejčastější problémy.

Vývoj hry proběhne v herním enginu Unity, blíže přiblíženém v kapitole \ref{sec:unity}, s pomocí programovacího jazyka C\# s důrazem na standardy OOP. Již od samotného začátku je cílem tvořit hru modulárním způsobem pro případné jednodušší rozšíření v budoucnu. Grafika videohry bude vytvořena za pomocí softwaru Blender. Hra bude taktéž multiplatformní, aby byla dostupná co nejširšímu spektru hráčů.