\section{Blender}

Blender je open-source 3D modelovací nástroj, který nabízí vývojářům a umělcům širokou škálu funkcí pro tvorbu 3D modelů, animací, vizualizací a interaktivních aplikací.

Jedná se o výkonný nástroj, který umožňuje vytvářet vysoko kvalitní 3D modely pomocí různých technik, včetně polygonálních modelů, NURBS, sculptingu a dalších. Vývojáři mohou také vytvářet animace pomocí keyframe animace, simulací fyziky, ragdoll animace a dalších technik.

Blender poskytuje řadu nástrojů pro tvorbu textur a materiálů, včetně podpory pro PBR materiály, texturování procedurálními mapami a dalšími technikami. Vývojáři také mohou využívat různé osvětlovací techniky, včetně výpočtu globálního osvětlení (GI), ray tracingu a dalších. \cite{blender}

Také umožňuje vývojářům používat různé pluginy a rozšíření funkčnosti nástroje. Blender má také rozsáhlou komunitu, která nabízí různé nástroje, šablony a tutoriály pro vývojáře.
\\\\
\subsection{Způsob užití v projektu}

Software Blender byl využit pro tvorbu některé 3D grafiky. Původním plánem bylo realizovat veškerou 3D grafiku samostatně. Již v rané fázi projektu bylo zřejmé, že komplexní tvorba veškeré grafiky by měla za následek výrazné zpomalení vývoje a tedy i fakt, že by se některé funkce do alfa verze vůbec nedostaly. Z tohoto důvodu byl pro většinu grafiky využit asset pack zaměřený na low-poly kancelářské modely \cite{unity_asset}.