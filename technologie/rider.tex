\section{JetBrains Rider}

JetBrains Rider je IDE speciálně navržené pro vývoj aplikací v jazyce C\# a .NET. Jedná se o multiplatformní IDE, které je k dispozici pro všechny nejběžnější operační systémy.
Rider nabízí širokou škálu funkcí, včetně funkce refaktorování, inteligentního dokončování kódu, ladění aplikací a integrace s nástroji jako je Git. \cite{rider}

Rider také poskytuje podporu pro různé typy projektů, včetně projektů pro ASP.NET, Unity, Xamarin a .NET Core. IDE je také kompatibilní s různými databázovými systémy, jako jsou Microsoft SQL Server, MySQL a PostgreSQL.
V Rideru jsou také k dispozici nástroje pro testování kódu a prohlížení výsledků testování.

\subsection{Způsob užití v projektu}

Při vývoji bylo IDE Rider propojeno s herním enginem Unity. Posloužilo pro psaní veškerého kódu projektu.

Toto IDE bylo pro práci zvoleno z důvodu vysoké integrace s herním engine Unity. Příkladem by mohly být ukazatele odkazování metod na vnější objekty v enginu. Díky tomu lze jednodušeji optimalizovat a refaktorovat kód. K optimalizaci kódu taktéž přispívá doporučení pro privatizaci veřejných proměnných, které by mohly být privátními.

Rider má taktéž plugin pro Unity s názvem RiderFlow \cite{riderflow}. Plugin umožňuje ladit aplikaci přímo v Unity editoru, což je užitečnou funkcí při ladění chyb v herním engine. Bohužel nebylo možné tento plugin využít, protože v momentální verzi je plugin velmi nestabilní a nespolehlivý.
