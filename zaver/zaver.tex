Výsledným produktem je cílená beta verze projektu, která obsahuje všechny základní prvky, které jsou potřebné pro testování a další vývoj.

Sekce učení již podle předběžných testů úspěšně seznamuje hráče s probíranými tématy. Zároveň byl dodržen modulární postup řešení. Neměl by tedy být problém přidat další témata nebo upravit již existující.

Sekce náhodného scénáře dokresluje výše zmíněnou edukativní sekci. Zároveň je taktéž poměrně zajímavá ze svého technického hlediska, díky využité generaci prostředí.

Jedním z plánovaných prvků, ze kterých bylo bohužel potřeba v průběhu vývoje upustit, byla kompletní tvorba autorské grafiky. Nebylo tak učiněno z důvodu náročnosti, ale především s ohledem na časový rámec projektu. Do budoucna by však bylo vhodné vytvořit vlastní grafiku, která by mohla do jisté míry předefinovat celkový dojem z grafické stylizace.

V aktuální době je raná beta verze hry spustitelná na adrese \url{https://martas1293.itch.io/outside-the-net}.

Do budoucna je plánováno uzavřené testovaní hry. Toto testování bude mít za cíl získat zpětnou vazbu od hráčů, která bude sloužit k úpravám a vylepšení hry. Zároveň bude sloužit k ověření funkčnosti hry, identifikaci chyb a analýze výkonu. Dle přeběžných odhadů by se mělo testování zúčastnit několik málo desítek uživatelů.